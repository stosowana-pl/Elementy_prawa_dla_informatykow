\documentclass[12pt,a4paper]{article}
% !TEX TS-program = pdflatex
% !TEX encoding = UTF-8 Unicode

% This is a simple template for a LaTeX document using the "article" class.
% See "book", "report", "letter" for other types of document.


\usepackage{tabularx}

\usepackage[utf8]{inputenc} % set input encoding (not needed with XeLaTeX)
\usepackage{mathtools}
%% Examples of Article customizations
% These packages are optional, depending whether you want the features they provide.
% See the LaTeX Companion or other references for full information.

%% PAGE DIMENSIONS
\usepackage{geometry} % to change the page dimensions
\geometry{a4paper} % or letterpaper (US) or a5paper or....
% \geometry{margin=2in} % for example, change the margins to 2 inches all round
% \geometry{landscape} % set up the page for landscape
%   read geometry.pdf for detailed page layout information

\usepackage{graphicx} % support the \includegraphics command and options
\usepackage{amsfonts}
% \usepackage[parfill]{parskip} % Activate to begin paragraphs with an empty line rather than an indent

%% PACKAGES
\usepackage{listings}
%\usepackage{color}

\usepackage[OT4]{fontenc}
\usepackage[polish]{babel}
\usepackage{booktabs} % for much better looking tables
\usepackage{array} % for better arrays (eg matrices) in maths
\usepackage{paralist} % very flexible & customisable lists (eg. enumerate/itemize, etc.)
\usepackage{verbatim} % adds environment for commenting out blocks of text & for better verbatim
\usepackage{subfig} % make it possible to include more than one captioned figure/table in a single float
% These packages are all incorporated in the memoir class to one degree or another...

%% HEADERS & FOOTERS
\usepackage{fancyhdr} % This should be set AFTER setting up the page geometry
\pagestyle{fancy} % options: empty , plain , fancy
\renewcommand{\headrulewidth}{0pt} % customise the layout...
\lhead{}\chead{}\rhead{}
\lfoot{}\cfoot{\thepage}\rfoot{}

%% SECTION TITLE APPEARANCE
\usepackage{sectsty}
\allsectionsfont{\sffamily\mdseries\upshape} % (See the fntguide.pdf for font help)
% (This matches ConTeXt defaults)

%% ToC (table of contents) APPEARANCE
\usepackage[nottoc,notlof,notlot]{tocbibind} % Put the bibliography in the ToC
\usepackage[titles,subfigure]{tocloft} % Alter the style of the Table of Contents
\renewcommand{\cftsecfont}{\rmfamily\mdseries\upshape}
\renewcommand{\cftsecpagefont}{\rmfamily\mdseries\upshape} % No bold!

%opening
\title{Elementy prawa dla informatyków - wykład 3}
\author{}
\date{21.10.2014}
\begin{document}

\maketitle

\setcounter{section}{3}


\subsubsection{Utwory zależne}
\begin{itemize}
\item charakter zezwala na wykonywanie praw zależnych - prawo majątkowe czy osobiste
\item forma zezwolenia
	\begin{itemize}
	\item przeniesienie prawa majątkowego
	\end{itemize}
\item ...
\end{itemize}

\subsubsection{Wyłączone spod ochrony (art. 4. p.a)}
\begin{enumerate}
\item Akty normatywne lub ich urzędowe projekty
\item Urzędowe dokumenty, materiały, znaki i symbole
\item Opublikowane opisy patentowe lub ochronne
\item Proste informacje prasowe 
\end{enumerate}

\subsubsection{Podmiot prawa autorskiego}
\begin{itemize}
\item Twórca
	\begin{itemize}
	\item podmiot pierwotnie uprawniony
		\begin{itemize}
			\item prawa osobiste
			\item prawa majątkowe
		\end{itemize}
	\item domniemanie co do osoby twórcy
	\end{itemize}
\item Współtwórcy 
	\begin{itemize}
	\item domniemywa się, że wielkości udziałów są równe, każdy ze współtwórców może żądać określenie wielkości udziałów przez sąd, na podstawie wkładów pracy twórczej
	\item każdy ze współtwórców może wykorzystywać prawo autorskie do swojej części utworu mającej samodzielne znaczenie
	\item do wykorzystywania prawa autorskiego  do całości utworu potrzebna jest zgoda wszystkich współtwórców. W przypadku braku takiej zgody każdy ze współtwórców może żądać rozstrzygnięcia przez sąd, który orzeka uwzględniając interesy wszystkich współtwórców.
	\item każdy ze współtwórców może dochodzić roszczeń z tytułu naruszenia prawa autorskiego do całości utworu. Uzyskane świadczenie przypada wszystkim współtwórcom, stosownie do wielkości ich udziałów.
	\item do autorskich praw majątkowych przysługujących współtwórcom stosuje się odpowiednio przepisy Kodeksu Cywilnego o współwłasności w częściach ułamkowych
	\end{itemize}
\end{itemize}
\subsubsection{Utwory połączone}
\begin{itemize}
\item Artykuł 10. Jeżeli twórcy połączyli swoje ...
\end{itemize}
\subsubsection{Utwór pracowniczy}
\begin{enumerate}
\item Jeżeli ustawa  umowa o pracę nie stanowi inaczej, pracodawca, którego pracownik stworzył utwór  w wyniku wykonywana obowiązków ze stosunku pracy, nabywa z chwilą przyjęcia utworu autorskie prawa majątkowe w granicach wynikających z celu umowy o pracę i zgodnego zamiaru stron.
\item Jeżeli pracodawca, w okresie 2 lat od daty przyjęcia utworu nie przystąpi do rozpowszechniania utworu przeznaczonego w umowie o pracę do rozpowszechniania, twórca może wyznaczyć pracodawcy na piśmie odpowiedni termin na rozpowszechnienie utworu z tym skutkiem, że po jego bezskutecznym upływie  prawa uzyskane przez ...
\item Jeżeli umowa o pracę nie stanowi inaczej ...
\end{enumerate}


\subsubsection{Artykuł 74, ustęp 3} 
 Prawa majątkowe o programu komputerowego stworzonego przez pracownika w wyniku wykonywania obowiązków ze stosunku pracy przysługują pracodawcy, o ile umowa nie stanowi inaczej 



\subsubsection{Autorskie prawa majątkowe}
\begin{itemize}
\item Koncepcje monistyczne (np. Niemcy)
	\begin{itemize}
		\item prawo autorskie to prawo podmioowe zawierające zarówno elementy majątkowe, jak i nie majątkowe, niemożliwe jest ich rozdzielenie
			\begin{itemize}
				\item niezbywalność prawa (inter vivos) - możliwość ustanawiania uprawnień do korzystania z dzieła w oznaczonym zakresie
			\end{itemize}
	\end{itemize}
\item Koncepcja dualistyczna
	\begin{itemize}
		\item zbywalne prawa majątkowe
			\begin{itemize}
				\item ich czas trwania jest ograniczony
			\end{itemize}
		\item niezbywalne prawa majątkowe
	\end{itemize}
\item Różnice ze względów praktycznych zacierają się:
	\begin{itemize}
		\item w systemach dualistycznych istnieją mechanizmy utrudniające całkowite wyzbycie się praw
	\end{itemize}
\item Możliwość nabycia pierwotnych praw majątkowych przez inny podmiot niż twórca (pracodawca); twórcy przysługują ex lege tylko prawa osobiste
	\begin{itemize}
		\item niemożliwe w systemie monistycznym
	\end{itemize}
\item Ale, nie można przenieść ...
\end{itemize}

\subsubsection{Modele konstrukcji majątkowych praw autorskich}
\begin{itemize}
\item założenie, że prawa majątkowe zbliżone są do prawa własności
	\begin{itemize}
		\item prawo podmiotowe skuteczne erga omnes (dla wszystkich), wyłączne prawo dysponowania prawem
	\end{itemize}
\item założenie, że istnieje ,,monopol autorski''
\item skutki przyjęcia określonego założenia:
	\begin{description}
		\item[koncepcja własnościowa] - prawo jest nieograniczone, granicą jest jedynie ustawa
		\item[koncepcja monopolu ustawowego] - przysługują uprawnienia określone w ustawie, uprawnienia wyrażane nie przyznawane przez ustawę nie są objęte monopolem
			\begin{itemize}
				\item występuje w przypadku patentu
				\item systemy copyright' granicą monopolu jest interes ogółu (korzystanie poza granicą monopolu jest dopuszczalne, koncepcja fair-use)
			\end{itemize}
	\end{description}
\end{itemize}
Systemy prawa koncepcji własnościowej i monopolu autorskiego obecnie zbliżają się - umowy międzynarodowe (TRIPS, USA jest stroną konwencji berneńskiej, globalizacja).

\subsubsection{Artykuł 17 ustawy o prawie autorskim} 
,,Jeżeli ...

\subsubsection{Określone sposoby korzystania z dzieła (pola eksploatacji)}
,,Pole eksploatacji'' - każdy sposób korzystania z dzieła samodzielny pod względem technicznym (i najczęściej ekonomicznym). Wymienienie poszczególnych pól eksploatacji ....

\subsubsection{Wyłączne prawo do korzystania z dzieła}
Rozporządzanie nim na wszystkich polach eksploatacji? Prawo do wynagrodzenia...

\end{document}
