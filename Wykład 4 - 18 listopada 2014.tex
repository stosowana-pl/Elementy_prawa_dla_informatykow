\documentclass[12pt,a4paper]{article}
% !TEX TS-program = pdflatex
% !TEX encoding = UTF-8 Unicode

% This is a simple template for a LaTeX document using the "article" class.
% See "book", "report", "letter" for other types of document.


\usepackage{tabularx}

\usepackage[utf8]{inputenc} % set input encoding (not needed with XeLaTeX)
\usepackage{mathtools}
%% Examples of Article customizations
% These packages are optional, depending whether you want the features they provide.
% See the LaTeX Companion or other references for full information.

%% PAGE DIMENSIONS
\usepackage{geometry} % to change the page dimensions
\geometry{a4paper} % or letterpaper (US) or a5paper or....
% \geometry{margin=2in} % for example, change the margins to 2 inches all round
% \geometry{landscape} % set up the page for landscape
%   read geometry.pdf for detailed page layout information

\usepackage{graphicx} % support the \includegraphics command and options
\usepackage{amsfonts}
% \usepackage[parfill]{parskip} % Activate to begin paragraphs with an empty line rather than an indent

%% PACKAGES
\usepackage{listings}
%\usepackage{color}

\usepackage[OT4]{fontenc}
\usepackage[polish]{babel}
\usepackage{booktabs} % for much better looking tables
\usepackage{array} % for better arrays (eg matrices) in maths
\usepackage{paralist} % very flexible & customisable lists (eg. enumerate/itemize, etc.)
\usepackage{verbatim} % adds environment for commenting out blocks of text & for better verbatim
\usepackage{subfig} % make it possible to include more than one captioned figure/table in a single float
% These packages are all incorporated in the memoir class to one degree or another...

%% HEADERS & FOOTERS
\usepackage{fancyhdr} % This should be set AFTER setting up the page geometry
\pagestyle{fancy} % options: empty , plain , fancy
\renewcommand{\headrulewidth}{0pt} % customise the layout...
\lhead{}\chead{}\rhead{}
\lfoot{}\cfoot{\thepage}\rfoot{}

%% SECTION TITLE APPEARANCE
\usepackage{sectsty}
\allsectionsfont{\sffamily\mdseries\upshape} % (See the fntguide.pdf for font help)
% (This matches ConTeXt defaults)

%% ToC (table of contents) APPEARANCE
\usepackage[nottoc,notlof,notlot]{tocbibind} % Put the bibliography in the ToC
\usepackage[titles,subfigure]{tocloft} % Alter the style of the Table of Contents
\renewcommand{\cftsecfont}{\rmfamily\mdseries\upshape}
\renewcommand{\cftsecpagefont}{\rmfamily\mdseries\upshape} % No bold!

%opening
\title{Elementy prawa dla informatyków - wykład 4}
\author{}
\date{18.11.2014}
\setcounter{section}{4}
\begin{document}
\maketitle
\subsubsection{Dozwolony użytek}
Kiedy możemy  korzystać z utworu bez zgody autora
\subsubsection{Program komputerowy jako przedmiot prawa autorskiego}
\begin{itemize}
\item Dyrektywa Parlamentu Europejskiego i Rady Europejskiej 2009/24/WE z dnia 23 kwietnia 2009 w sprawie ochrony prawnej programów komputerowych (Dz. Urz. UE. L2009, Nr 111 str 16) 
\item eur-lex.europa.eu
\item Dyrektywa Rady Unii Europejskiej Nr 9/250/EWG z 14 maja 1991 o ochronie ...
\item Artykuł 278 paragraf 1 kodeksu karnego...
\item Artykuł 74 ustęp 1 prawa autorskiego
\begin{itemize}
	\item ,,Programy komputerowe podlegają ochronie jak utwory literackie, o ile przepisy niniejszego rozdziału nie stanowią inaczej''
	\item cel - możliwość zastosowania konwencji berneńskiej o ochronie dzieł literackich i artystycznych
	\item faktyczne odmienności między ochroną dzieł literackich i programów komputerowych
\end{itemize}
\item Warunek uznania programu za utwór - ,,przejaw działalności twórczej o indywidualnym charakterze''
	\begin{itemize}
		\item Funkcjonalny cel programów komputerowych ogranicza swobodę twórczą
	\end{itemize}
\item Ochrona niezależna od sposobu wyrażenia (kod źródłowy, kod maszynowy, znajdujący się w notatkach, dokumentacji projektowej, użytkowej)
	\begin{itemize}
		\item Dokumentacja jako taka nie jest uważana za program komputerowy (jest to ,,utwór literacki'')
	\end{itemize}
\end{itemize}
Koncepcje, pomysły i idee nie są chronione.
\begin{center}
\begin{tabularx}{\textwidth}{|X|X|}
\hline
\textbf{Elementy chronione} & \textbf{Elementy niechronione}\\
\hline
Konkretny ciąg instrukcji (kod źródłowy), elementy tekstowe programu & Algorytm, język programowania (elementy pozatekstowe) \\
\hline
Formę utworu (sposób wyrażenia) & treść (idea, pomysł)\\
\hline
\end{tabularx}
\end{center}
\subsubsection{Sprawa SAS Institute vs World Programming Entity}
Sygnatura sprawy - C/406/10

\subsubsection{Problem skutków zakresu ochrony}
\begin{itemize}
\item Wykluczenie ochrony logiki, układu, struktury programu umożliwia wykorzystywanie czyjegoś dorobku
	\subitem * Argument zarówno za, jak i przeciw
\item ochrona układów, struktury programu, procesów wykorzystywanych przez program zbliżałaby się do ochrony patentowej
\end{itemize}

Podmiot autorskich praw majątkowych do programów komputerowych:
\begin{itemize}
\item twórca
\item pracodawca
\end{itemize}

\subsubsection{Treść autorskich praw majątkowych do programów komputerowych}
\begin{itemize}
	\item Artykuł 74 ustęp 4 - autorskie prawo majątkowe do programu komputerowego z zastrzeżeniem przepisów 75 ust. 2 i 3 obejmują prawo do: (prawo do zmian w programie to prawa majątkowe, można je sprzedać)
		\begin{enumerate}
			\item trwałego lub czasowego zwielokrotnienia programu komputerowego w całości lub części jakimikolwiek środkami i w jakiejkolwiek formie; w zakresie, w którym dla wprowadzenia, wyświetlenia, stosowania, przekazywania; przechowywania programu komputerowego niezbędne jest jego zwielokrotnienie, czynności te wymagają zgody uprawnionego
			\item tłumaczenia, przystosowywania, zmiany układu lub jakichkolwiek innych zmian w programie komputerowym, z zachowaniem praw osoby, który tych zmian dokonała
			\item rozpowszechniania, w tym użyczenia lub najmu programu komputerowego lub jego kopii
		\end{enumerate}
\end{itemize}
\subsubsection{Zakres szerszy niż w przypadku innych utworów}
\begin{itemize}
\item wyłączne prawo wprowadzania zmian
\item określenie granic dekompilacji i wykorzystywania jej wyników
\item wykluczenie dozwolonego użytku osobistego
\item wykluczenie ...
\end{itemize}

\subsubsection{Artykuł 75 ustęp 1 p.a}
Jeżeli umowa nie stanowi inaczej, czynności wymienione w artykule 74 ustęp 4 punkt 1 (zwielokrotnienie) i 2 (tłumaczenie, przystosowanie, zmiany) nie wymagają zgody uprawnionego, jeżeli są niezbędne do korzystania z programu komputerowego. Możliwość zmiany, przystosowywania, tłumaczenia, zwielokrotnienia obejmuje przykładowo:
\begin{itemize}
\item ,,zwykłe'' korzystanie z programu
\item poprawianie błędów
\item testowanie programów (np.  żeby sprawdzić, czy nie ma wirusów)
\item zmienianie programu w zakresie dostosowywania do nowych wymagań
\end{itemize} 

\subsubsection{Artykuł 75 ustęp 2 p.a}
Nie wymaga zezwolenia uprawnionego:
\begin{enumerate}
\item Sporządzanie kopii zapasowej, jeżeli jest to niezbędne do korzystania z programu komputerowego 
\item Obserwowanie, badanie i testowanie programu komputerowego w celu poznania jego idei i zasad przez osobę posiadającą prawo do korzystania z egzemplarza programu komputerowego ...
\item zwielokrotnienie i ...
\end{enumerate}

Artykuł 75 ustęp 3 p.a o którym mowa w ustępie 2 punkt nie mogą być:
\begin{enumerate}
\item wykorzystywania  do innych celów niż osiągnięcie współdziałania niezależnie ...
\end{enumerate}

Dozwolony użytek z programu komputerowych:
\begin{itemize}
\item kopia zapasowa
\item black box analysis
\end{itemize}

\subsubsection{Dekompilacja, art. 75 ust. 2 pkt. 3  art. 75 ust. 3 p.a}
Określenie warunków, w których dopuszczalna jest dekompilacja:
\begin{itemize}
\item wykonywanie przez osobę uprawnioną
\item nie ma łatwo dostępnego innego źródła informacji
\item jedynym celem jest wykorzystanie rezultatów do osiągnięcia współdziałania ...
\end{itemize}
\subsubsection{Artykuł 29 p.a}
Wyjątek wśród przepisów o dozwolonych użytku (osobistym i publicznym) - prawo cytatu

\paragraph{}
Artykuł 29.1 - Wolno przytaczać w utworach stanowiących samoistną całość urywki rozpowszechnionych utworów lub dobrane utwory w całości ...

Poza zakresem dozwolonego użytku:
\begin{itemize}
\item użytek osobisty (art 23. p.a)
\item przejściowe lub incydentalne zwielokrotnienie (np. streaming filmu przez internet)
\item wykorzystywanie przez instytucje naukowe, biblioteki, ośrodki informacji
\item wykorzystywanie dla dobra niepełnosprawnych, korzystanie dla celów publicznych ...
\end{itemize}

(dozwolony użytek dla programów komputerowych poza prawem cytatu jest wykluczony)

\subsubsection{Autorskie prawa osobiste}
\begin{itemize}
\item prawo do autorstwa
\item prawo do oznaczania utworu swoim nazwiskiem lub pseudonimem albo udostępnienia go anonimowo
\item wykluczone są inne prawa osobiste (art. 16 ust. 3-5 p.a) t.j do:
	\begin{itemize}
		\item nienaruszalności treści i formy utworu oraz jego rzetelnego wykorzystania
		\item decydowania o pierwszym udostępnieniu utworu publiczności
		\item nadzoru nad sposobem korzystania z utworu
	\end{itemize}
\end{itemize}

\subsubsection{Artykuł 58 p.a} 
Jeżeli publiczne udostępnienie utworu następuje w nieodpowiedniej formie albo ze zmianami, którym twórca mógłby słusznie się sprzeciwić, może on po bezskutecznym wezwaniu do zaniechania naruszania odstąpić od umowy lub ją wypowiedzieć. Twórcy przysługuje prawo do wynagrodzenia określonego umową.

\subsubsection{Artykuł 77}
Uprawniony może domagać się od użytkowników programu komputerowego zniszczenia posiadanych przez nich środków technicznych ... (praktyczne znaczenie nie jest zbyt wielkie.)
\end{document}