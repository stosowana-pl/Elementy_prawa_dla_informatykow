\documentclass[12pt,a4paper]{article}
% !TEX TS-program = pdflatex
% !TEX encoding = UTF-8 Unicode

% This is a simple template for a LaTeX document using the "article" class.
% See "book", "report", "letter" for other types of document.


\usepackage{tabularx}

\usepackage[utf8]{inputenc} % set input encoding (not needed with XeLaTeX)
\usepackage{mathtools}
%% Examples of Article customizations
% These packages are optional, depending whether you want the features they provide.
% See the LaTeX Companion or other references for full information.

%% PAGE DIMENSIONS
\usepackage{geometry} % to change the page dimensions
\geometry{a4paper} % or letterpaper (US) or a5paper or....
% \geometry{margin=2in} % for example, change the margins to 2 inches all round
% \geometry{landscape} % set up the page for landscape
%   read geometry.pdf for detailed page layout information

\usepackage{graphicx} % support the \includegraphics command and options
\usepackage{amsfonts}
% \usepackage[parfill]{parskip} % Activate to begin paragraphs with an empty line rather than an indent

%% PACKAGES
\usepackage{listings}
%\usepackage{color}

\usepackage[OT4]{fontenc}
\usepackage[polish]{babel}
\usepackage{booktabs} % for much better looking tables
\usepackage{array} % for better arrays (eg matrices) in maths
\usepackage{paralist} % very flexible & customisable lists (eg. enumerate/itemize, etc.)
\usepackage{verbatim} % adds environment for commenting out blocks of text & for better verbatim
\usepackage{subfig} % make it possible to include more than one captioned figure/table in a single float
% These packages are all incorporated in the memoir class to one degree or another...

%% HEADERS & FOOTERS
\usepackage{fancyhdr} % This should be set AFTER setting up the page geometry
\pagestyle{fancy} % options: empty , plain , fancy
\renewcommand{\headrulewidth}{0pt} % customise the layout...
\lhead{}\chead{}\rhead{}
\lfoot{}\cfoot{\thepage}\rfoot{}

%% SECTION TITLE APPEARANCE
\usepackage{sectsty}
\allsectionsfont{\sffamily\mdseries\upshape} % (See the fntguide.pdf for font help)
% (This matches ConTeXt defaults)

%% ToC (table of contents) APPEARANCE
\usepackage[nottoc,notlof,notlot]{tocbibind} % Put the bibliography in the ToC
\usepackage[titles,subfigure]{tocloft} % Alter the style of the Table of Contents
\renewcommand{\cftsecfont}{\rmfamily\mdseries\upshape}
\renewcommand{\cftsecpagefont}{\rmfamily\mdseries\upshape} % No bold!

%opening
\title{Elementy prawa dla informatyków - wykład 2}
\author{}
\date{14.10.2014}
\begin{document}

\maketitle

\subsubsection{Materiały?}
\begin{itemize}
\item Komentarz do artykułu 28
\item Janusz Barta, Ryszard Mankiewicz - komentarz do ustawy o prawie autorskim i prawach konkretnych (agh -> kwestura -> lex -> aut )
\end{itemize}


\subsubsection{Utwór}
\begin{itemize}
\item każdy przejaw działalności twórczej
	\begin{itemize}
	\item indywidualny charakter
	\item ustalony w jakiejkolwiek postaci
	\item niezależnie od wartości, przeznaczenia i sposobu wyrażenia
	\end{itemize}
\item pomimo powyższego liczy się efekt, a nie przebieg procesu tworzenia
\item brak domniemania twórczego efektu działalności człowieka
\item konwencja berneńska $\rightarrow$ ochrona końcowego przedmiotu działalności, istotne jest to co zrobimy	
\item luźna definicja utworu - ,,Coś co zrobimy i ma wyraźne cechy indywidualnośći''
\end{itemize}

\subsubsection{Co musimy wykazać w sądzie, gdy ktoś pogwałci nasze prawa autorskie}
\begin{itemize}
\item że coś jest utworem
\item że mamy do niego prawa
\item że prawa zostały naruszone
\end{itemize}

Typy utworów:
\begin{itemize}
\item  wyrażane słowem, symbolami matematycznymi, znakami graficznymi, programy komputerowe
\item plastyczne, fotograficzne, lutnicze, wzornictwa przemysłowego, architektoniczne \textbackslash urbanistycnze, muzyczne\textbackslash słowno-muzyczne, sceniczne, pantomimiczne, audiowizualne
\end{itemize}

\subsubsection{Patenty}
\begin{itemize}
\item Programu nie można opatentować. Algorytm tak, ale musi mieć charakter przemysłowy?
\item Można opatentować wynalazek, którego integralną częścią jest program komputerowy $\rightarrow$ musi to być techniczny wynalazek?
\end{itemize}

\subsubsection{Kontrowersje}
\begin{itemize}
\item elementy chronionych utworów muzycznych
\item utwory kartograficzne
\item fotografie
\item utwory architektoniczne
	\begin{itemize}
		\item aspekty funkcjonalne, techniczne, materiałowe, konstrukcyjne, prawne
		\item problem konfrontacji praw architekta i właściciela budynku (dotyczy raczej budynków przynoszących większe zyski)
	\end{itemize}
\item utwory reklamowe
	\begin{itemize}
		\item  problem granic ochrony prawno-naukowej
		\item ustawa o zwalczaniu nieuczciwej konkurencji
	\end{itemize}
\end{itemize}

\subsubsection{Rodzaje utworów}
\begin{itemize}
\item Samoistne - ,,w pełni'' nieinspirowane cudzym dziełem
\item Samoistne inspirowane
\item Niesamoistne
	\begin{itemize}
		\item utwory współautorskie (programy)
		\item utwory zbiorowe
		\item utwory zależne (zależą od twórcy utworu pierwotnego, należy wymienić jego i tytuł utworu)
			\begin{itemize}
				\item adaptacje
				\item opracowania
				\item przeróbki
			\end{itemize}
		\item utwory z zapożyczeniami (przekraczającymi dozwolony użytek) 
	\end{itemize}
\end{itemize}

Problem pomiędzy utworem inspirowanym a zależnym.

Element twórczy - powstaje nowy utwór (nie powstaje utwór zależny np. w przypadku drobnych przeróbek, korekt)

Rozporządzanie i korzystanie z opracowania zależy od zezwolenia twórcy utworu pierwotnego (chyba, że wygasły)

Zezwolenie nie jest potrzebne na sporządzenie opracowania
\end{document}
